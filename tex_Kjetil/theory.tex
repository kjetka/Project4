

Ising model

Markhow chain - convergance
Error
Random number ?????

The theory and method sections are based on chapter 12 and 13 in Jensen, \cite{Jensen}.
\subsection{The Ising model}

The Ising model is a model used to simulate magnetic phase transitions of solids. In this project a somewhat simplified version of the model will be used, assuming no external magnetic field and a finite, 2 dimentional system. It is also assumed that the each spin can only take the values $ s  = \pm 1$.  In this model only the nearest neighbours affect each other, excluding long range effects. The energy in a system of a total of N spins is then defined as

\begin{equation}\label{eq: Energy_Ising}
E = -J \sum\limits_{\langle jk \rangle }^{N} s_ks_l
\end{equation}

with J being a coupling constant and  $ \langle jk \rangle $ indicating that the sum is over the nearest neighbours only. The useful quantity Energy per spin is defined as $ E_{spin}  = \frac{E}{N}$. 


\subsubsection{Statistical physics in the Ising model}

The spins in the Ising model follows Boltzmann statistics, meaning that the probability of a state $\ket{i}$ is defined as 
\begin{equation}\label{eq:boltzmann}
P(E_i) = \frac{e^{-E_i\beta}}{Z_{\beta}} 
\end{equation}

with the partition function $ Z_{\beta} = \int dE \ e^{-E\beta} $ normalizes the expression and $ \beta = (k_BT)^{-1} $. The partition function used in the project is discrete,$ Z_{\beta} = \sum\limits_{i}^{N} \ e^{-E_i\beta} $. As the temperature T increases, the probability of each state decreases, giving a broader distribution of probable states.  

In order to characterize the system, the mean energy, mean magnetization and mean absolute magnetization are important. The macroscopic property of mean energy $ \langle E \rangle $ is needed to define the heat capacity $ C_V$ of the system, while the microscopic effect of mean magnetization and the magnetic moment leads to the susceptibility $ \chi $. These are defined below: 
\begin{align}
\langle E \rangle &= \frac{1}{Z_{\beta}} \sum\limits_{i}^N E_i P_{\beta}(E_i)\\
\langle M \rangle &= \frac{1}{Z_{\beta}} \sum\limits_{i}^N M_i P_{\beta}(E_i)\\
\langle |M | \rangle &= \frac{1}{Z_{\beta}} \sum\limits_{i}^N |M|_i P_{\beta}(E_i)\\
 C_V &= \frac{1}{k_B T^2} \left( 	\langle E^2 \rangle - \langle E\rangle^2 	\right)\\
\chi &= \frac{1}{k_B T} \left( 	\langle M^2 \rangle - \langle M\rangle^2 	\right)
\end{align}


\subsubsection{Periodic boundary conditions}

At the boundaries of a finite spin matrix it is fewer nearest neighbours than in the bulk of the matrix. This is analogous to a surface of a material. By assuming periodic boundary conditions, the effects of the surface is neglected and easy to handle. For a 1 dimensional case with N spins, the neighbours of spin $ S_N $ is $ S_{N-1} $ and $ S_1 $. 

\subsection{Phase transitions}

A phase transition happens that a thermodynamically stable state of a system changes abruptly  when one or more thermodynamical  variables describing the structure reaches a critical value. In addition to changing state, macroscopic properties of the system must change. Melting of a solid is a example of an everyday phase transition, depending on a critical pressure and a critical temperature. At a critical temperature ($ T_C $) the Ising model undergoes a second order phase transition, where both the mean energy and magnetization is changed. 


A first order phase transition, like melting ice, have two state that coexist at the critical point, with a finite  correlation length
A second order phase transition is characterized by having 


Correlation length, forklare second order ????
spin correlation -related to suscepti


Regardless of the matrix size of the 2 dimensional system, there are only 16 



Energy, magnetic moment, susceptibility, heat capacity, critical temperature:


\begin{equation}\label{eq:critical_T}
T_C(L) - T_C(L=\infty) = a L^{-1/\nu}
\end{equation}
for matrixes of size $ L\times L $

\subsection{Simple example of the Ising model}

It is possible to model the $ 2\times 2 $ Ising model with periodic boundary conditions analytically. This specific system has $ N = 2^{L^2} = 2^4=16 $ different micro states.

To do:  beskrive alle data: tabell,  Eavg Mavg, Z, Cv, X, 


\[
Z = \sum_i^M e^{-\beta E_i} = e^{-\beta 8 J} + e^{-\beta 8 J} + e^{\beta 8 J}e^{\beta 8 J} + 12
\]

\[
= 4 \cosh\left( \beta 8 J \right) + 12
\]


Energy:
\[
\left< E \right> = k_B T^2 \left(\frac{\partial Z}{\partial T}
\right)_{V,N}
\]
\[
= k_B T^2 \frac{\partial}{\partial T} \left[\ln \left(4\cosh \left(\frac{8J}{k_BT}\right) +12\right) \right]\]

\[
\frac{\partial \ln Z}{\partial T} = \frac{\partial Z}{\partial \beta}\frac{\partial \beta}{\partial T} = \frac{\partial \ln Z}{\partial \beta}\left(\frac{-1}{k_B T^2}\right)
\]

\[
\left< E\right> = -\left(\frac{\partial Z}{\partial \beta} \right)_{V,N} = - \frac{\partial}{\partial \beta} \ln \left[ 4 \cosh \left( 8J\beta\right)+12\right]
\]

\[
= \frac{-1}{4\cosh (8J\beta) + 12}4 \sinh(8J\beta)8J\beta
\]
$
= \frac{-8J \sinh(8J\beta)}{3\cosh(J\beta)+ 4}
$

Following the same method, we found that:

\[
\left< |M| \right> = \frac{1}{Z} \sum_i^M M_i e^{\beta E_i}  = \frac{(8J)^2 \cosh(8J\beta )}{\cosh (8J\beta ) + 3}
\]

\[
\left< M \right> = 0
\]

\[
\left< E^2 \right> = \frac{8 \left( e^{8J\beta } + 1\right) }{\cosh (8J\beta ) + 3}
\]


\[
\left< M^2 \right> = \frac{1}{Z} \left( \sum_i^M M_i^2 e^{\beta E_i}\right) = \frac{2 \left( e^{8J\beta } + 2\right) }{\cosh (8J\beta ) + 3}
\]

We can use these to calculate the rest:

\[
C_V = k \beta^2\left( \left< E^2\right> - \left< E\right>^2 \right)
\]

\[
\chi = \beta \left( \left< M^2\right> - \left< M\right>^2\right)
\]

\begin{table}[H]
	\caption{text}
	\label{tab: makro}
\begin{tabular}{cccccc}
	No spin up  & Deg & Energy 	& Magnetization \\	\hline
0			&	1	&	-8J		&	-4		\\				
1			&	4	&	0		&	-2		\\
2			&	4	&	0		&	0		\\
2			&	2	&	8J		&	0		\\
3			&	4	&	0		&	2		\\
4			&	1	&	-8J		&	4		\\

\end{tabular}
\end{table}

See table \ref{tab: makro} for all the possible micro states the L=2 system can take. 


