The metropolis algorithm for determining the probability of a jump in a Monte Carlo cycle is a descent method. This report shows that it is not completely accurate, but the simplicity and speed it offers for complicated systems makes up for that. Together with parallel computing, the overall time of the program was manageable.  As the algorithm also allows for spin transitions that increase the overall energy, thus fluctuating around equilibrium, there is a limit to the precision of the produced data. 

With increasing temperature, thus increasing the average thermal energy, resulted in increased mean energy and decrease of mean magnetization. This also affected the probability distribution of the different energies available at equilibrium, from a step function at T = 1 to a distribution looking like the normal probability distribution. In addition, the Ising model shows a clear phase transition around a critical temperature, which is dependent on the size of the spin matrix. This affected the total magnetisation, by suddenly dropping towards 0 at $ T_C $.

When calculating the behaviour around the critical temperature, the highest temperature resolution should be used over the entire temperature range. It would also be a good addition to include larger spin matrices, as the estimate of the $ T_C(L=\infty) $ probably would be more accurate. Another method which could have improved this slightly is to average the 5-10 temperatures with the highest $ C_V $. However, this would not necessary give an increase in precision.
