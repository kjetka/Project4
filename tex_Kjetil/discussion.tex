



\subsection{The L=2 case}

The mean magnetic moment is fluctuating more than the mean absolute magnetic moment, see figure \ref{fig:l2magmagabs}. A small system  is strongly dependent on the periodic boundary conditions, meaning that for the L=2 system, three out of four spins determine each $ E_i $, linking them strongly together. This again means that if one spin changes, this very quickly have consequences for all the other spins. As this effect is expected to be lower at larger lattices, the smear out results of $ \langle \abs{M}\rangle$ is prefered to $ \langle {M}\rangle$ for calculating the magnetic susceptibility. 

Figure \ref{fig:l2energy} shows that the $ \langle E \rangle $ stabilizes at a lower value than the analytically calculated value, an effect also seen by the behaviour of $ \chi  $ and $ C_V $. The plots show the clear trend of the statistical development of the system to the most likely state, but it also shows that  Monte Carlo methods will not give exact answers. However the error is in the fourth digit and give a good approximation of the analytical value. Due to the random acceptance of less likely states, fluctuation is also seen.

It is also clear that the microscopic property of $ \langle M \rangle  $ is correlated to $\langle E \rangle  $, as expected from equation \ref{eq: Energy_Ising}.

\subsection{The L=20 system}

From the plots of the mean magnetization and energy for different temperatures, figures \ref{fig:her}, it is clear that the equilibration time is not identical for different temperatures. This  is probably partly due to the fact that higher temperatures fluctuates more, both while converging towards equilibrium and at the equilibrium. In order to be safe, a equilibration time of $ 2\E{5} $, corresponding to that of the highest temperature, and $ T = 2.4 $. This time corresponds to $ 0.2\cdot 10^6 $ Monte Carlo cycles, an important number when computing in parallel the way we chose to. As each thread calculated $ 10^6 $  Monte Carlo cycles, the equilibration time corresponds to only $ 20\% $ of the total amount of calculations. This means that the majority of each thread is at equilibrium when collecting the different parallel simulations.

As the temperature increases, the likelihood that the system will change  energy at equilibrium increases. This is due to inherent properties of the Boltzmann distribution. As the temperature increases, the difference between $ e^{-\beta E_i} $ and $ e^{-\beta E_j} $ decreases, resulting in higher variance for high temperatures. In real life, this corresponds to that the average thermal energy of each particle increases. Figure \ref{fig:L20_mag_T_2-4} shows this behaviour as greater fluctuations in both $\langle E \rangle  $, and $\langle M \rangle  $ at higher temperatures. 
 

\subsection{Phase transition, Critical temperature and benchmarks}

When a phase transition occur, the properties of a system changes abruptly. By studying equation \ref{eq_CV_analy} and figure \ref{fig:4eenergy}, the gradient increases $ T = 2.3 $. In figure \ref{fig:4ecv} this is mapped as a peak. The temperature corresponding to the top of the peak is the critical temperature, where the properties changes quickest. Figure \ref{fig:4ecv} indicates that the larger L, the more abruptly the system changes properties. For smaller lattices, the change is more gradual. As the heat capacity diverges for an infinite lattice, sharper peaks for higher lattice sizes is expected. The same applies for $ \chi $. 

By increasing the temperature resolution around $ T_C $ in the experiment, the hope was to gain a more precise value for $ T_C(L = \infty) $. However, the results show a fluctuation in this region, disrupting the expected trend for $ T_C (L) $. Figures \ref{fig:4ecv} and \ref{fig:4ex} shows that for L=80 and L=100, the maximum is at a different temperature than what would be expected by a symmetric peak. 

This fluctuation is not only located around $ T_C $, but is expected to occur at all temperatures, as long as the temperature resolution is big enough. The temperature step, dT, is relatively large on the interval $ T = 2.1-2.2 $ and $ T = 2.4-2.6 $ and the values of $ C_V $ and $ \chi $ changes significantly, the fluctuating effect is negligible.

Both the mean energy and magnetization fluctuates at random around equilibrium. The scale of this fluctuation is related to the temperature, as figure \ref{fig:distribution} indicates. When running $ 10^6 $ Monte Carlo cycles and collecting only $ C_V $ and $ \chi $ at the last MC-cycle, no two runs at the same temperature would yield the same results.  
 
 Figure \ref{fig:linfit}  and table \ref{tab: T_C} is affected by this, failing to illustrate the expected trend of lower $ T_C $ for larger lattices. 
 
 
The CPU-timing shows a clear speedup, but not as large as expected. This is most likely caused by factors like the $MPI_reduce$ function. This function performs a lot more calculations with two threads as with only one thread. 