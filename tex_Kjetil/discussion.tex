



\subsection{The L=2 case}

Hvorfor $ \langle \abs{M} \rangle $ flukturere mer enn $ \langle {M} \rangle $?
	Se forelesningsnotat for kommentar + diskusjon!
\subsection{The L=20 system}






\subsubsection{Equilibrium time  for the random L=20 system}



\subsubsection{Probability distrubition  for the L=20 system}



\subsection{Phase transition and Critical temperature}



benchmark (speed),

OBS: Plot of E, M, Cv, X as functions of T (put L as legend and plot together)

OBS: Indication of phase transition? (Peak - at least for Cv and X)

OBS: Use Equation \ref{eq:critical_T} to extract $T_C$.


Timing parallellisering





Phase transition
As the temperature is increased, the Boltzmann distribution predicts that more states will become stable, lowering the energy necessary to change state of each spin. This can be seen  in figure \ref{fig:4emag} from the fact that $ \langle M \rangle $ approaches 0 after $ T_C $