In this project we tried out many new concepts in our algorithm.

\begin{lstlisting}[language=C++]
for(i =0; i<temperature.size();i++){
   T = temperature[i];
   for(MC = 1; MC < MCcycles.size(); MC++){
       for(i=0; i< L*L; i++){
           ix, iy = random(i);
           // with random spin generator        
           Matrix(ix, iy) *= -1;
           dE = dEs[i];
           // dEs = vec({})
           MetropolisAlgorithm();
           // decide if the flip is accepted
           if(flip is accepted){
               Energy += dE;
               Magnetic += dM;
           }
      }
      //Add the new values to the sum of the values:
      mean_E += Energy;
      mean_E2 += Energy*Energy;
      mean_M += Magnetic;
      sum M\^2 += M\^2;
      sum |M| += |M|;
   }
//Before print, the values are divided by the number of Monte Carlo cycles to find the mean values.
}
\end{lstlisting}


\subsection{Monte Carlo cycles}

In Monte Carlo methods, the goal is to compute as many possible outcomes of the systems properties and find the mean value of them. By computing enough times, a steady state will be reached.


\subsection{Metropolis algorithm}

The metropolis algorithm is an algorithm that takes the system to the steady state. We want to find out what the real state of the system is when the outer parameters are what they are, for example temperature.

- Calculate total energy of initial lattice, $E_{tot}$

- Pick a random spin in the lattice

- Flip the spin

- Calculate the change in energy, $\Delta E$ (only five possibilities)

- If $\Delta E \leq 0$ - accept because we want to move to a state with the lowest energy

- If $\Delta E > 0$ - calculate $\omega = e^{-\beta \Delta E}$

- Compare $\omega$ with a random number $r$, if $r \leq \omega$ - accept new configuration

- Update mean values

- Repeat  

Should show how to find the five $\Delta E$s.

\subsection{Random number generator}

\subsection{Parallelizing}

\subsection{Unit tests}

Check the L=2 result with the analytical one
 - We did it visually

Make a small matrix (ordered initial), calculate energy - flip one spin, calculate energy - is the change in energy what we expect? 

 - Should have had it underveis
 - Kan neste gang lage tester som gjør at når vi implementerer nye ting som for eksempel classes eller paralelleisering, vet at det ikke har skjedd noe galt.




Metropolis (T,A,...)
	Stochastic matrix  - convergences (forhold eigenvalue).
	
	Hvilken random number engine
	
MPI:	
	
- Develop codes locally, run with some few processes and test your codes. Do benchmarking, timing and so forth on local nodes, for example your laptop or PC.
- When you are convinced that your codes run correctly, you can start your production runs on available supercomputers.
	
	
MPI functions:
%
%MPI_InitMPI_Init - initiate an MPI computation
%
%MPI_FinalizeMPI_Finalize - terminate the MPI computation and clean up
%
%MPI_Comm_sizeMPI_Comm_size - how many processes participate in a given MPI communicator?
%
%MPI_Comm_rankMPI_Comm_rank - which one am I? (A number between 0 and size-1.)
%
%MPI_SendMPI_Send - send a message to a particular process within an MPI communicator
%
%MPI_RecvMPI_Recv - receive a message from a particular process within an MPI communicator
%
%MPI_reduceMPI_reduce or MPI_AllreduceMPI_Allreduce, send and receive messages