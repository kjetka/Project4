\subsection{Ising model}

The two dimensional Ising model is a statistical model that allows us to investigate the temperature dependence of different properties of a magnet. The model consists of a two dimensional lattice of spins that can be in two different states, spin up, $\uparrow$, or spin down, $\downarrow$ \cite{Ising}. In our system we do not have an applied field, $B_a = 0$.

The energy of our system is then:

\begin{equation}\label{eq:total_energy}
E = - J \sum^N_{\left< kl \right>} s_k s_l
\end{equation}

where in our system $J > 0$, giving a ferromagnetic ordering, and $s_k, \, s_l = \pm 1$.


\subsection{Introduction to statistics}

The methods in this project require some statistics. Equation \ref{eq:probability_density} shows the probability density and in the Ising model, the probability distribution is Boltzmann distribution (Equation \ref{eq:Boltzmann}) where $Z$ is the partition function that normalizes the probability.

\begin{equation}\label{eq:probability_density}
P(a\leq X \leq b) = \int_b^a\, p(x)\, dx
\end{equation}

\begin{equation}\label{eq:Boltzmann}
P_i(\beta) = \frac{e^{-\beta E_i}}{Z}
\end{equation}

\begin{equation}\label{eq:Partitionfunction}
Z = \sum_{i=1}^M e^{-\beta E_i}
\end{equation}



\begin{equation}\label{eq:expectation_value}
\left< h \right>_X = \int \,h(x) \,p(x)\,dx
\end{equation}

\begin{equation}\label{eq:moment}
\left< x^n \right> = \int \,x^n \,p(x)\,dx
\end{equation}

\begin{equation}\label{eq:mean_value}
\left< x \right> = \int \,x \,p(x)\,dx
\end{equation}

Partition function

Boltzman

\subsection{Magnetic properties}

Energy, magnetic moment, susceptibility, heat capacity.

\begin{equation}\label{eq:critical_T}
T_C(L) - T_C(L=\infty) = a L^{-1/\nu}
\end{equation}

Markhow chain - convergance

\subsection{Analytical solutions for L=2}

L=2 case:
\begin{table}[H]
	\caption{text}
	\label{tab: makro}
\begin{tabular}{cccccc}
	No spin up  & Deg & Energy 	& Magnetization \\	\hline
0			&	1	&	-8J		&	-4		\\				
1			&	4	&	0		&	-2		\\
2			&	4	&	0		&	0		\\
2			&	2	&	8J		&	0		\\
3			&	4	&	0		&	2		\\
4			&	1	&	-8J		&	4		\\

\end{tabular}
\end{table}

Error
Random number

The partition function:
\[
Z = \sum_i^M e^{-\beta E_i} = e^{-\beta 8 J} + e^{-\beta 8 J} + e^{\beta 8 J}e^{\beta 8 J} + 12
\]
\[
= 2e^{-\beta 8 J}+ 2e^{\beta 8 J} + 12 = 4\left(\frac{e^{-\beta 8 J}+ e^{-\beta 8 J}}{2}\right)+12
\]
\[
= 4 \cosh\left( \beta 8 J \right) + 12
\]
 The energy:
\[
\left< E \right> = k_B T^2 \left(\frac{\partial Z}{\partial T}
\right)_{V,N}
\]
\[
= k_B T^2 \frac{\partial}{\partial T} \left[\ln \left(4\cosh \left(\frac{8J}{k_BT}\right) +12\right) \right]\]

\[
\frac{\partial \ln Z}{\partial T} = \frac{\partial Z}{\partial \beta}\frac{\partial \beta}{\partial T} = \frac{\partial \ln Z}{\partial \beta}\left(\frac{-1}{k_B T^2}\right)
\]

\[
\left< E\right> = -\left(\frac{\partial Z}{\partial \beta} \right)_{V,N} = - \frac{\partial}{\partial \beta} \ln \left[ 4 \cosh \left( 8J\beta\right)+12\right]
\]

\[
= \frac{-1}{4\cosh (8J\beta) + 12}4 \sinh(8J\beta)8J\beta
\]

\[
= \frac{-8J \sinh(8J\beta)}{3\cosh((J\beta)+ 4}
\]

Following the same method, we found that:

\[
\left< |M| \right> = \frac{1}{Z} \sum_i^M M_i e^{\beta E_i}  = \frac{(8J)^2 \cosh(8J\beta )}{\cosh (8J\beta ) + 3}
\]

\[
\left< M \right> = 0
\]

\[
\left< E^2 \right> = \frac{8 \left( e^{8J\beta } + 1\right) }{\cosh (8J\beta ) + 3}
\]


\[
\left< M^2 \right> = \frac{1}{Z} \left( \sum_i^M M_i^2 e^{\beta E_i}\right) = \frac{2 \left( e^{8J\beta } + 2\right) }{\cosh (8J\beta ) + 3}
\]

We can use these to calculate the rest:

\[
C_V = k \beta^2\left( \left< E^2\right> - \left< E\right>^2 \right)
\]

\[
\chi = \beta \left( \left< M^2\right> - \left< M\right>^2\right)
\]