The theory and method sections are based on chapter 12 and 13 in Jensen, \cite{Jensen}.

\subsection{Ising model}

The two dimensional Ising model is a statistical model that allows us to investigate the temperature dependence of different properties of a magnet. Our model consists of a two dimensional lattice of spins that can be in two different states, spin up, $\uparrow$, or spin down, $\downarrow$ \cite{Ising}. The number of spins in each dimension is $L$ and the lattice is assumed to be quadratic. In our system we do not have an applied field, $B_a = 0$. The energy of our system is then given by Equation \ref{eq:total_energy}.

\begin{equation}\label{eq:total_energy}
E = - J \sum^N_{\left< kl \right>} s_k s_l
\end{equation}

where in our system $J > 0$, $J$ is the coupling constant, giving a ferromagnetic ordering, and $s_k, \, s_l = \pm 1$, $s_k$ is the $k$-th spin of $N$ number of spins.

To get easier numbers to do calculation with we scaled $ e^{-\beta E} $ by setting $ T' = \frac{k_B T	}{J} $ and $ E' = JE_{kl} $, with $ E_{kl} = \sum\limits_{<k,l>} s_ks_l $ Then $e^{-\beta E} $ can be written as $ e^{-E'/T'} $.

\subsubsection{Periodic boundary conditions}

Because we did not have an infinite lattice, we included periodic boundary conditions. That means that when the spins on the edges are flipped, we use the spins on the other side of the lattice as nearest neighbors, to calculate the change in energy.

\subsection{Introduction to statistics}

The methods in this project require some statistics. Equation \ref{eq:probability_density} shows the probability density and in the Ising model, the probability distribution is Boltzmann distribution (Equation \ref{eq:Boltzmann}) where $Z$ is the partition function that normalizes the probability (Equation \ref{eq:Partitionfunction}). To calculate $Z$, we need to know all the different states,$i$, and their energy, $E_i$. It can be calculated for small systems (\ref{sec:analytical_sol} Analytical solutions for $L$=2), but it would be very time consuming to include it in Monte Carlo Sampling. A method that does not include it is necessary.

\begin{equation}\label{eq:probability_density}
P(a\leq X \leq b) = \int_b^a\, p(x)\, dx
\end{equation}

\begin{equation}\label{eq:Boltzmann}
P_i(\beta) = \frac{e^{-\beta E_i}}{Z}
\end{equation}

\begin{equation}\label{eq:Partitionfunction}
Z = \sum_{i=1}^M e^{-\beta E_i}
\end{equation}

In this project we are looking at different properties of a magnet. That is energy, magnetic moment, heat capacity and susceptibility. To get out this data form our Ising model, we need some statistical terms. We will use Monte Carlo cycles and the Metropolis algorithm to calculate the second order moment (Equation \ref{eq:moment}) and the first order moment also called the mean value (Equation \ref{eq:mean_value}) of the energy and the magnetic moment. These values, we can use to get the heat capacity and the susceptibility.

\begin{equation}\label{eq:moment}
\left< x^2 \right> = \int \,x^2 \,P(x)\,dx
\end{equation}

\begin{equation}\label{eq:mean_value}
\left< x \right> = \int \,x \,P(x)\,dx
\end{equation}

\subsection{Magnetic properties}

The energy of the system is found by using Equation \ref{eq:total_energy}. The total magnetic moment is found by adding the magnetic moments of all spins in the lattice. The susceptibility represents how a material responds to an applied magnetic field, $B_a$. The heat capacity is the ratio of heat added to a material and the resulting temperature change.

From thermodynamics we have these relations \cite{thermal}:
\begin{multicols}{2}
\[
\left< E \right> = k_B T^2 \frac{\partial \ln Z}{\partial T}
\]

\[
C_V = \frac{\partial \left< E \right>  }{\partial T}
\]

\[
\left< M \right> = k_B T \frac{\partial \ln Z}{\partial B_a}
\]

\[
\chi = \frac{\partial \left< M \right>  }{\partial B_a}
\]
\end{multicols}
We are looking at the temperature dependence of these properties and finding the critical temperature, $T_C$,  where the phase transition between paramagnetic and ferromagnetic occurs. When $T< T_C$ the system exhibit spontaneous magnetization it is ferromagnetic, but when $T > T_C$ the net magnetization is zero it is paramagnetic. This phase transition is of second order. That means that the correlation length of the system diverges when $T \rightarrow T_C$, it spans the whole system. The correlation length represents the correlation between the spins. It is a spacial unit because it indicates how far away spins can be from each other and still correlate. We can find out that the Ising model exhibits a second-order phase transition since the heat capacity diverges. The temperature dependence of the magnetic susceptibility from mean-field theory is shown in Equation \ref{eq:mag_suscept_T}.

\begin{equation}\label{eq:mag_suscept_T}
\chi \propto (T-T_C)^{-\gamma}
\end{equation}
where $\gamma$ is a critical exponent and is different for different temperatures.

The critical temperature for finite lattices is lattice dependent, the result for finite lattice sizes is related to the critical temperature for the infinite lattice by Equation \ref{eq:critical_T}.

\begin{equation}\label{eq:critical_T}
T_C(L) - T_C(L=\infty) = a L^{-1/\nu}
\end{equation}

\subsection{Analytical solutions for $L$=2}\label{sec:analytical_sol}

Here are the calculation of the analytical values for the system where $L$=2. These were used as benchmark calculations for the numerical calculations.

\begin{table}[H]
	\caption{This table lists the possible states and their accompanying energies when $L$=2 The full list of energies without degeneration is in Table \ref{tab:micro_energies}.}
	\label{tab:macro_energies}
\begin{tabular}{cccccc}
	No spin up  & Deg & Energy 	& Magnetization \\	\hline
0			&	1	&	-8J		&	-4		\\				
1			&	4	&	0		&	-2		\\
2			&	4	&	0		&	0		\\
2			&	2	&	8J		&	0		\\
3			&	4	&	0		&	2		\\
4			&	1	&	-8J		&	4		\\

\end{tabular}
\end{table}

First we calculated the partition function using Equation \ref{eq:total_energy} and the numbers in Table \ref{tab:macro_energies}:
\[
Z = \sum_i^N e^{-\beta E_i} = e^{-\beta 8 J} + e^{-\beta 8 J} + e^{\beta 8 J}+e^{\beta 8 J} + 12
\]
\[
= 2e^{-\beta 8 J}+ 2e^{\beta 8 J} + 12 = 4\left(\frac{e^{-\beta 8 J}+ e^{-\beta 8 J}}{2}\right)+12
\]
\[
= 4 \cosh\left( \beta 8 J \right) + 12
\]

Using thermodynamical relations we calculate the expectation values of the energy and magnetic moment\cite{thermal}.

\[
\left< E \right> = k_B T^2 \left(\frac{\partial Z}{\partial T}
\right)_{V,N}
\]

\[
= k_B T^2 \frac{\partial}{\partial T} \left[\ln \left(4\cosh \left(\frac{8J}{k_BT}\right) +12\right) \right]\]
\hrule
\[
\frac{\partial \ln Z}{\partial T} = \frac{\partial Z}{\partial \beta}\frac{\partial \beta}{\partial T} = \frac{\partial \ln Z}{\partial \beta}\left(\frac{-1}{k_B T^2}\right)
\]
\hrule
\[
\left< E\right> = -\left(\frac{\partial Z}{\partial \beta} \right)_{V,N} = - \frac{\partial}{\partial \beta} \ln \left[ 4 \cosh \left( 8J\beta\right)+12\right]
\]

\[
= \frac{-1}{4\cosh (8J\beta) + 12}4 \sinh(8J\beta)8J\beta
\]

\[
= \frac{-8J \sinh(8J\beta)}{3\cosh((J\beta)+ 4}
\]

Following the same method, we found that:
\[
\left< |M| \right> = \frac{1}{Z} \sum_i^N M_i e^{\beta E_i}  = \frac{(8J)^2 \cosh(8J\beta )}{\cosh (8J\beta ) + 3}
\]

\[
\left< M \right> = 0
\]

\[
\left< E^2 \right> = \frac{1}{Z} \left( \sum_i^N E_i^2 e^{\beta E_i}\right) =\frac{8 \left( e^{8J\beta } + 1\right) }{\cosh (8J\beta ) + 3}
\]


\[
\left< M^2 \right> = \frac{1}{Z} \left( \sum_i^N M_i^2 e^{\beta E_i}\right) = \frac{2 \left( e^{8J\beta } + 2\right) }{\cosh (8J\beta ) + 3}
\]

We can use these to calculate the rest, with these definitions:

\[
C_V = k \beta^2\left( \left< E^2\right> - \left< E\right>^2 \right)
\]

\[
\chi = \beta \left( \left< M^2\right> - \left< M\right>^2\right)
\]