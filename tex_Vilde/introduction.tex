The aim of this project is to use a model of a magnetic material to explore different new aspects of programming. Among them are random number generators, the Monte Carlo method, the Metropolis algorithm and parallelizing. These are all important tools when programming physical systems.

We want to simulate a two dimensional ferromagnetic material using the Ising model. We are using Monte Carlo sampling with the Metropolis algorithm to find the steady state of the system at different temperatures.We will calculate the mean values of important properties of a magnetic material, the energy, the magnetic moment, the susceptibility and the heat capacity. Finally, we will use the results at different temperatures to find the critical temperature where the material has a phase transition from ferromagnetic with a spontaneous magnetic moment, to a paramagnet with zero magnetic moment. 

This report contains some theory on the Ising model, statistics and properties of a magnetic material. It explains the methods used in the calculation and simulation. Furthermore, the results are presented together with a discussion of the result. Finally some concluding remarks are presented. 