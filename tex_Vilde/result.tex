\subsection{Matrix dimension $L$=2}

The result form the two dimensional lattice with lattice size $L$=2 and at T= 1.0 $\text{k}_\text{B}T/\text{J}$ are in Figures \ref{fig:L_2_energy}, \ref{fig:L_2_magnetic_abs}, \ref{fig:L_2_heat_capacity} and \ref{fig:L_2_susceptibility}. From the plots, we can see that the numerical result gives a good agreement with the analytical values calculated in \ref{sec:analytical_sol} Analytical solutions for L=2 after approximately $5\cdot 10^5$ Monte Carlo cycles. Table \ref{tab:compare_values} lists the result after $10^6$ Monte Carlo cycles, and it show a good agreement as well. 

\begin{table}\caption{This table compares the analytical values for $L$=2 with the numerical ones after $10^6$ Monte Carlo cycles. The values are in units per spin and at T=1.0 $\text{k}_\text{B}T/\text{J}$.}\label{tab:compare_values}
\begin{tabular}{ccc}
& Numerical: & Analytical:\\ \hline
$\left<E\right>\, [E_{kl}]$ &   -1.9958 & -1.9960\\
$\left<E^2\right>\, [E_{kl}^2]$ &   15.9664 & 15.9679\\
$\left<M\right>$ &    0.0451 & 0\\
$\left<M^2\right>$ &    3.9930 & 3.9933\\
$\left<|M|\right>$ &    0.9986 & 0.9987\\
$\chi \, [J/k_B^T]$ &   3.9849 & 3.9933\\
$C_V \, [J^2/k_B^3T^2]$& 0.0335 & 0.0321\\
\end{tabular}
\end{table}

\begin{figure}[H]
\includegraphics[width=\linewidth]{../results/4b/L_2_energy}\caption{This is a plot of the expectation value of the energy per spin verus number of Monte Carlo cycles. The plot shows that we have a good agreement after $ 5 \cdot 10^{5} $ MC cycles.}\label{fig:L_2_energy}
\end{figure}

\begin{figure}[H]
\includegraphics[width=\linewidth]{../results/4b/L_2_magnetic_abs}\caption{This is a plot of the expectation value of the mean absolute value of the magnetic moment per spin verus number of Monte Carlo cycles. The plot shows that we have a good agreement after $ 5 \cdot 10^{5} $ MC cycles.}\label{fig:L_2_magnetic_abs}
\end{figure}

\begin{figure}[H]
\includegraphics[width=\linewidth]{../results/4b/L_2_heat_capasity}\caption{This is a plot of the heat capacity per spin verus number of Monte Carlo cycles. The plot shows that we have a good agreement after $ 5 \cdot 10^{5} $ MC cycles.}\label{fig:L_2_heat_capacity}
\end{figure}

\begin{figure}[H]
\includegraphics[width=\linewidth]{../results/4b/L_2_susceptibility}\caption{This is a plot of the susceptibility per spin verus number of Monte Carlo cycles. The plot shows that we have a good agreement after $ 5 \cdot 10^{5} $ MC cycles.}\label{fig:L_2_susceptibility}
\end{figure}

\begin{figure}[H]
\includegraphics[width=\linewidth]{../results/4b/L_2_magnetic}\caption{This is a plot of the expectation value of the mean value of the magnetic moment per spin verus number of Monte Carlo cycles. The plot shows that we would not have had a good agreement after $ 5 \cdot 10^{5} $ MC cycles, when not using the absolute value as in Figure \ref{fig:L_2_magnetic_abs}.}\label{fig:L_2_magnetic}
\end{figure}


We see that the energy converges fastest. The magnetic moment is not converging as fast but still fast, we are plotting the absolute value, and that makes it converge faster, because the oscillation between the same size, but different signs does not show. Figure \ref{fig:L_2_magnetic} shows how the magnetic moment converges much slower. It can be shown however that the magnetic moment will converge, but slower and we would have needed more Monte Carlo cycles. 

\subsection{Matrix dimension $L$ = 20}

We increased the size of the system and 

HMM: Should define an area that is enough for equilibrium!

OBS: Need the number of MC cycles to reach equilibrium!

OBS: Need equilibration time! (5 1e5?)

OBS: Comment accepted configs T dependency

\subsubsection{Different temperatures}

\begin{figure}[H]
\includegraphics[width=\linewidth]{../results/4c/En_mag_T1_0}\caption{This is a plot of both the expectation value of the energy and absolute magnetic moment per spin verus number of Monte Carlo cycles at T = 1.0 $\text{k}_\text{B}T/\text{J}$. The plot shows that an equilibrium is reached already at $2 \cdot 10^{5}$ MC cycles.}\label{fig:L_20_energy_mag_T_1.0}
\end{figure}

\begin{figure}[H]
\includegraphics[width=\linewidth]{../results/4c/En_mag_T2_4}\caption{This is a plot of both the expectation value of the energy and absolute magnetic moment per spin verus number of Monte Carlo cycles at T = 2.4 $\text{k}_\text{B}T/\text{J}$. The plot shows that an equilibrium is reached at around $5 \cdot 10^{5}$ MC cycles.}\label{fig:L_20_energy_mag_T_2.4}
\end{figure}

\subsubsection{Initial state}

\begin{figure}[H]
\includegraphics[width=\linewidth]{../results/4c/ran_order_T1}\caption{This is a plot of both the expectation value of the energy and absolute magnetic moment per spin verus number of Monte Carlo cycles at T = 1.0 K. The plot shows the difference in the behaviour of the ordered initial state and a random initial state.}\label{fig:L_20_initial_T_1.0}
\end{figure}

\begin{figure}[H]
\includegraphics[width=\linewidth]{../results/4c/ran_order_T2}\caption{This is a plot of both the expectation value of the energy and absolute magnetic moment per spin verus number of Monte Carlo cycles at T = 2.4 K. The plot shows the difference in the behaviour of the ordered initial state and a random initial state.}\label{fig:L_20_initial_T_2.4}
\end{figure}

\subsubsection{Accepted configurations}

\begin{figure}[H]
\includegraphics[width=\linewidth]{../results/4c/L_20_total_accepted}\caption{This is a plot of the total number of accepted configurations versus number of Monte Carlo cycles with random initial state.}\label{fig:total_accepted}
\end{figure}

%\begin{figure}[H]
%\includegraphics[width=\linewidth]{../results/4c/L_20_accepted_configs}\caption{This is a plot of the percentage accepted of attempted configurations versus  Monte Carlo cycles with random initial state.}\label{fig:percentage_accepted}
%\end{figure}

\subsection{Energy probability}

OBS: Compare result with computed variance!

OBS: Discuss behavior (In Discussion - maybe just merge result and discussion?)

Computed variance (from same dataset?):

$$ \sigma_E^2 = \left< E^2\right> - \left< E\right>^2 $$
$$\text{FWHM} = 2 \sqrt{2ln2} \sigma \approx 2.355 \sigma$$ 

T = 1.0 K:
$$ \sigma_E^2 = 638181 - (-798.855)^2 = 11.69 $$
$$ \sigma = 3.42 $$
$$ \text{FWHM} \approx 2.355 \cdot 3.24 = 7.63 $$

T = 2.4 K:
$$ \sigma_E^2 =   247886 - (-494.628)^2 = 3229.14 $$
$$ \sigma = 56.8  $$
$$ \text{FWHM} \approx 2.355 \cdot 56.8 = 133.76 $$

\begin{figure}[H]
\includegraphics[width=\linewidth]{../results/4d/d_T_1probability}\caption{This is a plot of the energy probability when T = 1.0 K. The energy is the total energy of the 2D lattice with $20\times 20$ spins.}\label{fig:probability_T_1.0}
\end{figure}

\begin{figure}[H]
\includegraphics[width=\linewidth]{../results/4d/d_T_2_4probability}\caption{This is a plot of the energy probability when T = 2.4 K.The energy is the total energy of the 2D lattice with $20\times 20$ spins.}\label{fig:probability_T_2.4}
\end{figure}

\subsection{Increasing dimensionality/ Critical temperature}

\begin{figure}[H]
\includegraphics[width=\linewidth]{../results/4e/4e_energy}\caption{This is a plot if the energy versus temperature around the critical temperature for the different lattice sizes with $L$ = 40, $L$ = 60, $L$ = 80 and $L$ = 100.}\label{fig:4e_energy}
\end{figure}

\begin{figure}[H]
\includegraphics[width=\linewidth]{../results/4e/4e_mag}\caption{This is a plot if the absolute magnetic moment versus temperature around the critical temperature for the different lattice sizes with $L$ = 40, $L$ = 60, $L$ = 80 and $L$ = 100.}\label{fig:4e_magnetic}
\end{figure}

\begin{figure}[H]
\includegraphics[width=\linewidth]{../results/4e/4e_Cv}\caption{This is a plot if the heat capacity versus temperature around the critical temperature for the different lattice sizes with $L$ = 40, $L$ = 60, $L$ = 80 and $L$ = 100.}\label{fig:4e_heat_capa}
\end{figure}

\begin{figure}[H]
\includegraphics[width=\linewidth]{../results/4e/4e_x}\caption{This is a plot if the susceptibility versus temperature around the critical temperature for the different lattice sizes with $L$ = 40, $L$ = 60, $L$ = 80 and $L$ = 100. The exact value is $T_C =  kTC/J = 2/ \ln(1+\sqrt{
2}) \approx 2.269 \,\text{k}_\text{B} \,\text{K}$ \cite{Onsager}.}\label{fig:4e_suscept}
\end{figure}

OBS: Indication of phase transition? (Peak - at least for Cv and X)

OBS: Compare behaviour with equations?

OBS: Use Equation \ref{eq:critical_T} to extract $T_C$.

Getting these equations from \ref{eq:critical_T} where $\nu = 1$:
\begin{align*}
T_C(40) - T_C(\infty) &= a \cdot 40^{-1}\\
T_C(60) - T_C(\infty) &= a \cdot 60^{-1}\\
T_C(80) - T_C(\infty) &= a \cdot 80^{-1}\\
T_C(100) - T_C(\infty) &= a \cdot 100^{-1}\\
\end{align*}

(Sett inn tall!)
\begin{align}\label{eq:find_critial_T}
T_C(\infty) &= - a \cdot 40^{-1} + 2.28\\
T_C(\infty) &= - a \cdot 60^{-1} + 2.28\\
T_C(\infty) &= - a \cdot 80^{-1} + 2.28\\
T_C(\infty) &= - a \cdot 100^{-1} + 2.28\label{eq:find_critial_T_end}
\end{align}

\begin{figure}[H]
\includegraphics[width=\linewidth]{../results/4f/critical_t}\caption{This is a plot of Equation \ref{eq:critical_T} with different values of $L$ (See Equations \ref{eq:find_critial_T} - \ref{eq:find_critial_T_end}). The interssections represent the solution. They should have all had a cross section in the same place, and the y-value of the intersection would have been the critical temperature when $L \rightarrow \infty$.}\label{fig:critical_T}
\end{figure}

Exact $T_C =  kTC/J = 2/ \ln(1+\sqrt{
2}) \approx 2.269$ \cite{Onsager}

\begin{table}\caption{This is a list of the CPU time with different numbers of processes for $L$=60 and number of Monte Carlo cycles were $10^6$.}\label{tab:CPU}
\begin{tabular}{cc}
Number of processors:& CPU time [s]: \\ \hline
 1 & 513.069\\
 2 & 306.975\\
\end{tabular}
\end{table}