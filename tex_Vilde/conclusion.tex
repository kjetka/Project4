In this project we used a model of a magnetic material to explore different new aspects of programming. We used random number generators, the Monte Carlo method, the Metropolis algorithm and parallelizing.

We simulated a two dimensional ferromagnetic material using the Ising model. We used Monte Carlo sampling with the Metropolis algorithm to find the steady state of the system at different temperatures and calculated the mean values of important properties of a magnetic material, the energy, the magnetic moment, the susceptibility and the heat capacity. Finally, we used these results at different temperatures to find the critical temperature where the model had a second order phase transition from ferromagnetic with a spontaneous magnetic moment, to a paramagnet with zero magnetic moment. 

We made the program so that we could have run it on a supercomputer with more processors than our computers that has only two, but we chose not to do it. Running in a supercomputer with more processors than two, would have given us more data, which is important in statistical models. We also should have included unit tests in our programming, to help us during the process. 